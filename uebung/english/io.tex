\newpage

\section*{Input/Output}
\begin{aufgabe}[String formatting]
Start the Python interpreter.
\begin{auflistung}
\item Display \lstinline{float} numbers with a given number of positions before and after decimal point, and in exponential notation.
\item Display \lstinline{int} numbers in hexadecimal and octal notation.
\item Given a person's given name, family name, residence and age, display the data in one line, e.g.:
\begin{verbatim}
"James Kirk is 35 years old and lives in Iowa."
\end{verbatim}
\item Try more formatting possibilities: \texttt{\underline{http://docs.python.org/lib/typesseq-strings.html}}
\end{auflistung}
\end{aufgabe}

\begin{aufgabe}[Command line parameters]
Write a program which prints its name and all its command line parameters.
\end{aufgabe}

\begin{aufgabe}[Read files]
Write a program which counts the frequency of the word "spam" in a file.
\end{aufgabe}

\begin{aufgabe}[Read and write files]
Given a file in which each lines contains arbitrary many numbers, separated by colons:
\begin{verbatim}
13.0:14.5:17:0.03:-2
89:-0.2354:6666
\end{verbatim}
Read the numbers, add all numbers in one line and write the numbers into a new file in the same format as above, only that each line additionally contains the sum at the end.

\hinweis Look at the string methods \lstinline{strip}, \lstinline{split} and \lstinline{join}.
\end{aufgabe}


