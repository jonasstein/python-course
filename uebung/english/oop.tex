\section*{Object Oriented Programming}

\begin{aufgabe}[Point Class]
\begin{teilaufgabe}
Implement the Point class from the lecture. Implement a human-readable print output and ensure that two points are recognised as equal when their $x$ and $y$ values match.
\end{teilaufgabe}
\begin{teilaufgabe}
Implement the $+$ and $-$ operator for two points:
\begin{displaymath}
\left(\begin{array}{c}
x_1\\
y_1
\end{array}\right)
+
\left(\begin{array}{c}
x_2\\
y_12
\end{array}\right)
=
\left(\begin{array}{c}
x_1 + x_2\\
y_1 + y_2
\end{array}\right), \hspace{5mm}
\left(\begin{array}{c}
x_1\\
y_1
\end{array}\right)
*
\left(\begin{array}{c}
x_2\\
y_12
\end{array}\right)
= (x_1 * x_2) + (y_1 * y_2)
\end{displaymath}
The following should be possible:
\begin{lstlisting}
>>> print Point(1, 2) + Point(4, 2)
(5, 4)
>>> print Point(1, 2) * Point(4, 2)
8
\end{lstlisting}
\end{teilaufgabe}

\begin{teilaufgabe}
Implement the norm of a point with the help of the above scalar product:
\begin{displaymath}
\left\| p\right\| = \sqrt{p*p}
\end{displaymath}
\end{teilaufgabe}
\end{aufgabe}

\begin{aufgabe}[Bank account]
\begin{teilaufgabe}
Implement a bank account class with attributes for the account number, the account balance and the name of the account holder. The account must not be overdrawn; ensure that any access to the account balance does not result in a negative account balance.
\end{teilaufgabe}
\begin{teilaufgabe}
Additionally to the account data, add an attribute for the interest rate, which is the same for all accounts. Implement a methods which informs the user about the current interest rate, and another method which calculates the interest for the current account balance.
\end{teilaufgabe}
\end{aufgabe}

