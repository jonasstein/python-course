\section*{Input/Output}
\begin{aufgabe}[String-Formatierung]
Starten Sie den Python-Interpreter. 
\begin{auflistung}
\item Geben Sie \lstinline{float}-Zahlen mit festgelegter Anzahl an Vor- und Nachkommastellen aus und in Exponentialdarstellung.
\item Geben Sie \lstinline{int}-Zahlen in Hexadezimal- und Oktaldarstellung aus.
\item Sie wissen "uber eine Person die Daten \lstinline{vorname}, \lstinline{nachname}, \lstinline{wohnort}, \lstinline{alter}. Geben Sie die Daten so aus, dass Sie z.B. folgende Zeile erhalten: 
\begin{verbatim}
"James Kirk ist 35 Jahre alt und wohnt in Iowa."
\end{verbatim}
\item Probieren Sie weitere Formatierungen aus:\\ \texttt{\underline{http://docs.python.org/lib/typesseq-strings.html}}
\end{auflistung}
\end{aufgabe}

\begin{aufgabe}[Kommandozeilenparameter]
Schreiben Sie ein Programm, welches seinen Namen und alle ihm "ubergebenen Parameter ausgibt.
\end{aufgabe}

\begin{aufgabe}[Dateien lesen]
Schreiben Sie ein Programm, welches die H"aufigkeit des Wortes \glqq Spam\grqq{} in einer Datei z"ahlt. \hinweis{Es gibt eine hilfreiche String-Methode!}
\end{aufgabe}

\begin{aufgabe}[Dateien lesen und schreiben]
Gegeben sei eine Datei, in der in jeder Zeile beliebig viele, durch Doppelpunkt getrennte Zahlen stehen:
\begin{verbatim}
13.0:14.5:17:0.03:-2
89:-0.2354:6666
\end{verbatim}
Lesen Sie die Zahlen ein, addieren Sie alle Zahlen einer Reihe und geben Sie die Zahlen in einer neuen Datei aus, im selben Format wie oben, nur dass als letztes zus"atzlich die Summe steht.

\hinweis Schauen Sie sie sich die String-Methoden \lstinline{strip}, \lstinline{split} und \lstinline{join} an.
\end{aufgabe}


