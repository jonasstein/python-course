\section*{Pythons Standardbibliothek}

\begin{aufgabe}[Passwort-Generator]
Schreiben Sie ein Programm, welches ein acht Zeichen langes Passwort generiert. \hinweis Schauen Sie sich \lstinline{string.letters} und \lstinline{string.digits} an!
\end{aufgabe}

\begin{aufgabe}
Schreiben Sie ein Programm, welches alle Dateien eines Verzeichnisses mit Endung \texttt{htm} umbenennt, sodass die neue Endung \texttt{html} lautet.
\end{aufgabe}

\begin{aufgabe}[CSV]
Gegeben sei folgende Daten-Datei:
\begin{verbatim}
"Mr. Spock","Vulcan"
"Freddie ""Nightmare"" Krueger, Jr.","Elm Street 42,
Springfield" 
\end{verbatim}
(Der Zeilenumbruch im zweiten Datensatz ist gewollt!). Die Datens"atze bestehen aus zwei Feldern, Name und Adresse. Lesen sie alle Datens"atze ein und geben Sie sie auf dem Bildschirm aus.

\hinweis Diesen Datensatz m"ochte man nicht selbst parsen. (Warum?)
\end{aufgabe}

\begin{aufgabe}[Spa"s mit regul"aren Ausdr"ucken]
Schreiben Sie ein Programm, welches alle html-Tags aus einer html-Datei entfernt. (html-Tags sind in spitzen Klammern eingeschlossen, also z.B. \texttt{<body>}, \texttt{</body>}.)
\end{aufgabe}

\begin{aufgabe}[XML-RPC]
Unter der Adresse \lstinline{http://grow.zam.kfa-juelich.de:8000} l"auft ein XML-RPC-Server. Erstellen Sie einen XML-RPC-Client und nutzen Sie dessem Methoden \lstinline{listMethods} und {methodHelp} um herauszufinden, welche Methoden der Server anbietet und wie man sie benutzt.
\end{aufgabe}



