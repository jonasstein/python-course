\section*{Ausnahmen}
\begin{aufgabe}
Lesen Sie vom Benutzer eine Zahl ein und geben Sie ihr Quadrat aus. Was passiert, wenn der Benutzer etwas anderes als eine Zahl eingibt? Geben Sie eine benutzerfreundliche Fehlermeldung aus!
\end{aufgabe}

\begin{aufgabe}[Fakult"at erweitert]
Die Fakult"at ist nur f"ur nat"urliche Zahlen definiert. Wie reagiert die Funktion aus Aufgabe \ref{fakultaet} auf negative Parameter? Lassen Sie die Funktion in diesem Fall einen \texttt{ValueError} ausl"osen.
\end{aufgabe}

\begin{aufgabe}Schreiben Sie ein Programm, welches in einer Endlosschleife Eingaben vom Benutzer einliest und jeweils pr"uft, ob die Eingabe ein Palindrom ist (verwenden Sie die Funktion aus Aufgabe \ref{palindrom}). 

Was passiert, wenn der Benutzer \texttt{Ctrl-C} oder \texttt{Ctrl-D} dr"uckt? "Andern Sie das Programm, sodass der Benutzer beim Dr"ucken von \texttt{Ctrl-C} oder \texttt{Ctrl-D} gefragt wird, ob er das Programm beenden m"ochte.
\end{aufgabe}

\newpage

\begin{aufgabe}
Mit dem Modul \texttt{readline} k"onnen f"ur die Benutzereingaben mittels \texttt{raw\_input} Editierm"oglichkeiten wie in der Shell aktiviert werden. Dieses Modul ist standardm"a"sig nur unter *nix-Systemen verf"ugbar. Folgender Code importiert das Modul nicht auf Windows-Rechnern:
\begin{lstlisting}
if not sys.platform.startswith("win"):
    import readline
    import rlcompleter
    readline.parse_and_bind("tab: complete")
\end{lstlisting}
Warum ist ein Ansatz mit Ausnahmen besser? Schreiben Sie den Code so um, dass er mit Ausnahmen arbeitet.
\end{aufgabe}

