\section*{Statements}
\begin{aufgabe}[Schleifen und Verzweigungen]
Schreiben Sie ein Programm, welches zehn Zahlen von der Tastatur einliest und anschlie"send deren Summe ausgibt.

\hinweis \lstinline{int()} und \lstinline{float()} k"onnen auch auf Strings angewendet werden.
\begin{teilaufgabe}
Schreiben Sie das Programm so um, dass es abbricht, falls die Zwischensumme gr"o"ser als 42 ist.
\end{teilaufgabe}
\begin{teilaufgabe}
Schreiben Sie das obige Programm so um, dass es negative Zahleneingaben ignoriert.\end{teilaufgabe}
\begin{teilaufgabe}
Schreiben Sie das obige Programm so um, dass es solange Zahlen einliest, bis der Benutzer \texttt{ende} eintippt.
\end{teilaufgabe}
\end{aufgabe}

\begin{aufgabe}
Schreiben Sie ein Programm, welches zehn Strings von der Tastatur einliest und in einer Liste speichert. Anschlie"send sollen in allen Strings der Buchstabe \texttt{a} durch ein \texttt{e} ersetzt werden und das Ergebnis ausgegeben werden. 

\hinweis Benutzen Sie Listen, aber greifen Sie nicht mit Indices \lstinline{liste[i]} auf die einzelnen Elemente zu!
\end{aufgabe}

\begin{aufgabe}
Die \lstinline{ord}-Funktion liefert die ASCII-Darstellung eines Zeichens zur"uck. Schreiben Sie ein Programm, welches einen String in ASCII-Darstellung ausgibt.

\hinweis Auch hier werden keine Indices ben"otigt!
\end{aufgabe}

