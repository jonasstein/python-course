\section*{Zus"atzliche Aufgaben}

\begin{aufgabe}[UNO/Mau-Mau]
Implementieren Sie ein einfaches UNO-Spiel (oder Mau-Mau-Spiel) ohne Sonderregeln f"ur bestimmte Karten. Gehen Sie dabei wie folgt vor:

\begin{teilaufgabe}[Grundlagen]
Vorschlag f"ur eine Strukturierung in Klassen:
\begin{auflistung}
\item Klasse \lstinline{Karte} mit Attributen \lstinline{farbe} und \lstinline{wert} und einer Methode \lstinline{kompatibel(self, other)}, die "uberpr"uft, ob zwei Karten aufeinander gelegt werden k"onnen.
\item Klasse f"ur ein gemischtes Kartendeck, welche von \lstinline{list} erbt:
\begin{lstlisting}
class DeckUNO(list):
    WERTE = ["0", "1", "2", "3", "4", "5", "6", "7", "8", "9"]
    FARBEN = ["Rot", "Gelb", "Gruen", "Blau"]

    def __init__(self):
        karten = []

        for farbe in self.FARBEN:
            for wert in self.WERTE:
                karten.append(Karte(farbe, wert))

        random.shuffle(karten)

        list.__init__(self, karten)
\end{lstlisting}
\item Klasse \lstinline{Spieler}. Der init-Methode wird ein Kartendeck-Objekt "ubergeben, davon werden sechs Karten entfernt und als Handkarten gespeichert; weitere Parameter (Spielername) sind denkbar. Zus"atzlich ben"otigt der Spieler Methoden zum Durchf"uhren eines Spielzuges. M"ochte man verschiedene Typen von Spielern einsetzen (Computerspieler, menschlicher Spieler), unterscheiden sich diese Methoden. Man k"onnte dann die Klassen f"ur die verschiedenen Spieler von einer allgemeinen \lstinline{Spieler}-Klasse erben lassen.
\end{auflistung}
Lassen Sie das Programm so lange laufen, bis ein Spieler gewonnen hat oder keine Karten mehr vom Talon gezogen werden k"onnen. Daf"ur k"onnten eigene Ausnahmen n"utzlich sein, welche von der allgemeinen \lstinline{Exception} abgeleitet werden:
\begin{lstlisting}
class WinException(Exception):
    pass

class TalonEmptyException(Exception):
    pass
\end{lstlisting}
\end{teilaufgabe}
\begin{teilaufgabe}[Spezialkarten]
Noch Lust? Implementieren Sie die Sonderkarten f"ur  Aussetzen, Karten ziehen, Farbwunsch etc.
\end{teilaufgabe}
\begin{teilaufgabe}[F"ur Profis]
Die sortierte Kartenliste im Kartendeck kann einfacher erzeugt werden. Man kann eine List Comprehension verwenden, und auch das \lstinline{itertools}-Modul k"onnte n"utzlich sein\dots
\end{teilaufgabe}
\end{aufgabe}

\begin{aufgabe}[F"ur Profis]
Wie man einfache Klassen als beliebiege Strukturen verwenden kann, wurde in der Vorlesung gezeigt. Verbessern Sie das Prinzip und schreiben Sie eine Klasse \lstinline{Bunch}, deren \lstinline{init}-Methode beliebige Keyword-Argumente nimmt und diese in entsprechenden Attributen speichert:
\begin{lstlisting}
punkt = Bunch(x=2, y=3)
print punkt.x, punkt.y

person = Bunch(vorname="Homer", nachname="Simpson", telefon=123456)
print person.vorname, person.nachname, person.telefon
\end{lstlisting}
\end{aufgabe}


