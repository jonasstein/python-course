\section*{Datentypen I}

\begin{aufgabe}[Erste Schritte, Zahlen]
Starten Sie den Python-Interpreter.
\begin{auflistung}
\item Benutzen Sie die Online-Hilfe: \lstinline{help()}, \lstinline{help(complex)}, \lstinline{dir(complex)}
\item Machen Sie sich mit den Datentypen \texttt{int}, \texttt{float}, \texttt{complex} und den grundlegenden Rechenoperationen vertraut.
\item Was passiert, wenn Sie in einer Rechnung verschiedene Zahlen-Typen kombinieren?
\item Geben Sie $1/3$ als Dezimalzahl $0.333333$\dots aus.
\item Was ergibt $1/0$?
\item Arbeiten Sie mit Variablenzuweisungen. Was passiert, wenn Sie einer Variable nacheinander Objekte unterschiedlicher Datentypen zuweisen?
\end{auflistung}
\end{aufgabe}

\begin{aufgabe}[Strings]
\begin{teilaufgabe}
Schreiben Sie ein Programm, welches nacheinander den Vornamen und den Nachnamen des Benutzers abfragt. Anschlie"send soll der vollst"andige Name zusammen mit einer Begr"u"sung in einer Zeile ausgegeben werden, z.B.:
\begin{lstlisting}
Dein Vorname? Jim
Dein Nachname? Kirk
Langes Leben und Friede, Jim Kirk
\end{lstlisting}
\end{teilaufgabe}
\begin{teilaufgabe}
Erweitern Sie das obige Programm, sodass es zus"atzlich folgendes ausgibt:
\begin{auflistung}
\item Die Buchstaben 2 bis 5 vom Vornamen und den vorletzen Buchstaben des Nachnamens
\item Den vollst"andigen Namen in Gro"sbuchstaben
\item Ob der Nachname auf \glqq mann\grqq{}  endet
\item \dots ? Probieren Sie noch weitere String-Methoden aus:\\ \texttt{\underline{http://docs.python.org/lib/string-methods.html}}
\end{auflistung}
\end{teilaufgabe}
\end{aufgabe}

\newpage
\begin{aufgabe}[Listen]
Starten Sie den Python-Interpreter.
\begin{auflistung}
\item Legen Sie eine Liste mit mehreren Elementen an und greifen Sie auf verschiedene Listenelemente zu: Das zweite, die vierten bis sechten, das letzte. Was passiert, wenn Sie einen ung"ultigen Listenindex angeben?
\item F"ugen Sie der Liste nachtr"aglich Elemente hinzu, l"oschen Sie einzelne Elemente. Wie geht Python mit verschiedenen Datentypen in der Liste um?
\item F"ugen Sie zwei Listen zusammen
\item Kann man eine Liste als Element in eine Liste einf"ugen?
\end{auflistung}
\end{aufgabe}

