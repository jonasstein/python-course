\section{Einf"uhrung}

\begin{frame}{Was ist Python?}
\alert{Python:} dynamische Programmiersprache, welche verschiedene Programmierparadigmen unterst"utzt:
\begin{itemize}
\item prozedurale Programmierung
\item objektorientierte Programmierung
\item funktionale Programmierung
\end{itemize}
Standard: Python-Bytecode wird im Interpreter ausgef"uhrt ("ahnlich Java)\\
$\rightarrow$ \alert{plattformunabh"angiger Code}
\end{frame}

\begin{frame}{Warum Python?}
\begin{itemize}
\item Syntax ist klar, leicht zu lesen \& lernen (fast Pseudocode)
\item intuitive Objektorientierung
\item volle Modularit"at, hierarchische Pakete
\item Fehlerbehandlung mittels Ausnahmen
\item dynamische, \glqq High Level\grqq-Datentypen
\item umfangreiche Standard-Bibliothek f"ur viele Aufgaben
\item einfache Erweiterbarkeit durch C/C++, Wrappen von C/C++-Bibliotheken
\end{itemize}
\alert{Schwerpunkt: Programmiergeschwindigkeit!}
\end{frame}

\begin{frame}{Ist Python schnell genug?}
\begin{itemize}
\item f"ur rechenintensive Algorithmen: evtl. besser Fortran, C, C++ 
\item f"ur Anwenderprogramme: Python ist schnell genug!
\item Gro"steil der Python-Funktionen sind in C geschrieben
\item Performance-kritische Teile k"onnen jederzeit in C/C++ ausgelagert werden
\item erst analysieren, dann optimieren!
\end{itemize}
\end{frame}

\begin{frame}[fragile]{Hallo Welt!}
\begin{lstlisting}[style=Python]
#!/usr/bin/env python

# This is a commentary
print "Hello world!"
\end{lstlisting}
\begin{lstlisting}[style=Shell]
$ python hello_world.py
Hello world!
$
\end{lstlisting}%$
\begin{lstlisting}[style=Shell]
$ chmod 755 hello_world.py
$ ./hello_world.py
Hello world!
$
\end{lstlisting} %$
\end{frame}

\begin{frame}[fragile]{Hallo User}
\begin{lstlisting}[style=Python]
#!/usr/bin/env python

name = raw_input("What's your name? ")
print "Hello", name
\end{lstlisting}
\begin{lstlisting}[style=Shell]
$ ./hello_user.py
What's your name? Rebecca
Hello Rebecca
$
\end{lstlisting}
\end{frame}

\begin{frame}{Starke und dynamische Typisierung}
\alert{Starke Typisierung:}
\begin{itemize}
\item Objekt ist genau von einem Typ! String ist immer String, \texttt{int} immer \texttt{int}
\item Gegenbeispiele: PHP, JavaScript, C: \texttt{char} kann als \texttt{short} betrachtet werden, \texttt{void~*} kann alles sein
\end{itemize}
\alert{Dynamische Typisierung: }
\begin{itemize}
\item keine Variablendeklaration
\item Variablennamen k"onnen nacheinander unterschiedliche Datentypen zugewiesen werden
\item Erst zur Laufzeit werden Eigenschaften eines Objekts untersucht
\end{itemize}
\end{frame}

\begin{frame}[fragile]{Starke und dynamische Typisierung}
\begin{lstlisting}[style=Python]
number = 3
print number, type(number)
print number + 42
number = "3"
print number, type(number)
print number + 42
\end{lstlisting}
\begin{lstlisting}[style=Shell]
3 <type 'int'>
45
3 <type 'str'>
Traceback (most recent call last):
  File "test.py", line 6, in ?
    print number + 42
TypeError: cannot concatenate 'str' and 
'int' objects
\end{lstlisting}
\end{frame}

\begin{frame}[fragile]{Interaktiver Modus}
Der Interpreter kann im interaktiven Modus gestartet werden:
\begin{lstlisting}[style=Shell]
$ python
Python 2.6 (r26:66714, Feb  3 2009, 20:52:03) 
[GCC 4.3.2] on linux2
Type "help", "copyright", "credits" or ...
>>> print "hello world"
hello world
>>> a = 3 + 4
>>> print a
7
>>> 3 + 4
7
>>>
\end{lstlisting} %$
\end{frame}

\begin{frame}{Dokumentation}
Online-Hilfe im Interpreter:
\begin{itemize}
\item \alert{\lstinline{help()}}: allgemeine Hilfe zu Python
\item \alert{\lstinline{help(obj)}}: Hilfe zu einem Objekt, z.B. einer Funktion oder einem Modul
\item \alert{\lstinline{dir()}}: alle belegten Namen 
\item \alert{\lstinline{dir(obj)}}: alle Attribute eines Objekts
\end{itemize}
\vspace{5mm}
Offizielle Dokumentation: \href{http://docs.python.org/}{http://docs.python.org/}
\end{frame}

\begin{frame}[fragile]{Dokumentation}
\begin{lstlisting}[style=Shell]
>>> help(dir)
Help on built-in function dir:
...
>>> a = 3
>>> dir()
['__builtins__', '__doc__', '__file__', 
'__name__', 'a']
>>> help(a)
Help on int object:
...
\end{lstlisting}
\end{frame}

