\section{Einf"uhrung}

\begin{frame}
\frametitle{Was ist Python?}
\alert{Python:} dynamische Programmiersprache, welche verschiedene Programmierparadigmen unterst"utzt:
\begin{itemize}
\item prozedurale Programmierung
\item objektorientierte Programmierung
\item funktionale Programmierung
\end{itemize}
Standard: Python-Bytecode wird im Interpreter ausgef"uhrt ("ahnlich Java)\\
$\rightarrow$ \alert{plattformunabh"angiger Code}
\end{frame}

\begin{frame}
\frametitle{Warum Python?}
\begin{itemize}
\item Syntax ist klar, leicht zu lesen \& lernen (fast Pseudocode)
\item intuitive Objektorientierung
\item volle Modularit"at, hierarchische Pakete
\item Fehlerbehandlung mittels Ausnahmen
\item dynamische, \glqq High Level\grqq-Datentypen
\item umfangreiche Standard-Bibliothek f"ur viele Aufgaben
\item einfache Erweiterbarkeit durch C/C++, Wrappen von C/C++-Bibliotheken
\end{itemize}
\alert{Schwerpunkt: Programmiergeschwindigkeit!}
\end{frame}

\begin{frame}
\frametitle{Ist Python schnell genug?}
\begin{itemize}
\item f"ur rechenintensive Algorithmen: evtl. besser Fortran, C, C++ 
\item f"ur Anwenderprogramme: Python ist schnell genug!
\item Gro"steil der Python-Funktionen sind in C geschrieben
\item Performance-kritische Teile k"onnen jederzeit in C/C++ ausgelagert werden
\item erst analysieren, dann optimieren!
\end{itemize}
\end{frame}

\begin{frame}[fragile]
\frametitle{Hallo Welt!}
\begin{lstlisting}[style=Python]
#!/usr/bin/env python

# Dies ist ein Kommentar
print "Hallo Welt!"
\end{lstlisting}
\begin{lstlisting}[style=Shell]
$ python hallo_welt.py
Hallo Welt!
$
\end{lstlisting}%$
\begin{lstlisting}[style=Shell]
$ chmod 755 hallo_welt.py
$ ./hallo_welt.py
Hallo Welt!
$
\end{lstlisting} %$
\end{frame}

\begin{frame}[fragile]
\frametitle{Hallo User}
\begin{lstlisting}[style=Python]
#!/usr/bin/env python

name = raw_input("Wie heisst du?")
print "Hallo", name
\end{lstlisting}
\begin{lstlisting}[style=Shell]
$ ./hallo_user.py
Wie heisst du?
Rebecca
Hallo Rebecca
$
\end{lstlisting}
\end{frame}

\begin{frame}
\frametitle{Starke und dynamische Typisierung}
\alert{Starke Typisierung:}
\begin{itemize}
\item Objekt ist genau von einem Typ! String ist immer String, \texttt{int} immer \texttt{int}
\item Gegenbeispiel C: \texttt{char} kann als \texttt{short} betrachtet werden, \texttt{void~*} kann alles sein
\end{itemize}
\alert{Dynamische Typisierung: }
\begin{itemize}
\item keine Variablendeklaration
\item Variablennamen k"onnen nacheinander unterschiedliche Datentypen zugewiesen werden
\item Erst zur Laufzeit werden Eigenschaften eines Objekts untersucht
\end{itemize}
\end{frame}

\begin{frame}[fragile]
\frametitle{Starke und dynamische Typisierung}
\begin{lstlisting}[style=Python]
zahl = 3
print zahl, type(zahl)
print zahl + 42
zahl = "Rebecca"
print zahl, type(zahl)
print zahl + 42
\end{lstlisting}
\begin{lstlisting}[style=Shell]
3 <type 'int'>
45
Rebecca <type 'str'>
Traceback (most recent call last):
  File "test.py", line 6, in ?
    print zahl + 42
TypeError: cannot concatenate 'str' and 
'int' objects
\end{lstlisting}
\end{frame}

\begin{frame}[fragile]
\frametitle{Interaktiver Modus}
Der Interpreter kann im interaktiven Modus gestartet werden:
\begin{lstlisting}[style=Shell]
$ python
Python 2.4.1 (#1, Oct 13 2006, 16:58:04)
[GCC 4.0.2 20050901 (prerelease) ...
Type "help", "copyright", "credits" or ...
>>> print "hallo welt"
hallo welt
>>> a = 3 + 4
>>> print a
7
>>> 3 + 4
7
>>>
\end{lstlisting} %$
\end{frame}

\begin{frame}
\frametitle{Dokumentation}
Online-Hilfe im Interpreter:
\begin{itemize}
\item \lstinline{help()}: allgemeine Hilfe zu Python
\item \lstinline{help(obj)}: Hilfe zu einem Objekt, z.B. einer Funktion oder einem Modul
\item \lstinline{dir()}: alle belegten Namen 
\item \lstinline{dir(obj)}: alle Attribute eines Objekts
\end{itemize}
\vspace{5mm}
Offizielle Dokumentation: \alert{\underline{\texttt{http://docs.python.org/}}}\\
\vspace{5mm}
Dive into Python: \alert{\underline{\texttt{http://diveintopython.org/}}}

\end{frame}

\begin{frame}[fragile]
\frametitle{Dokumentation}
\begin{lstlisting}[style=Shell]
>>> help(dir)
Help on built-in function dir:
...
>>> a = 3
>>> dir()
['__builtins__', '__doc__', '__file__', 
'__name__', 'a']
>>> help(a)
Help on int object:
...
\end{lstlisting}
\end{frame}

%%% Local Variables: 
%%% mode: latex
%%% latex-run-command: pdflatex
%%% TeX-master: "vortrag"
%%% End: 
