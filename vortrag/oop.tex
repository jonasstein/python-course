\section{Objektorientierte Programmierung}

\begin{frame}{Objektorientierte Programmierung}
\begin{itemize}
\item Bisher: \alert{prozedurale Programmierung}
\begin{itemize}
\item Daten
\item Funktionen, die Daten als Parameter entgegennehmen und Ergebnis zur"uckliefern
\end{itemize}
\item Alternative: Fasse zusammengeh"orige Daten und Funktionen zusammen zu \alert{eigenen Datentypen}
\item $\rightarrow$ Erweiterung von Strukturen/Datenverb"unden aus C/Fortran
\end{itemize}
\end{frame}

\begin{frame}[fragile]{Einfache Klassen als Structs verwenden}
\begin{lstlisting}[style=Python]
class Point:
    pass

p = Point()
p.x = 2.0
p.y = 3.3
\end{lstlisting}
\begin{itemize}
\item \alert{Klasse}: Eigener Datentyp (hier: \texttt{Point})
\item \alert{Objekt}: Instanz der Klasse (hier: \texttt{p})
\item Attribute (hier \texttt{x}, \texttt{y})  k"onnen dynamisch hinzugef"ugt werden
\end{itemize}
\end{frame}

\begin{frame}[fragile]{Klassen}
\begin{lstlisting}[style=Python]
class Point:
    def __init__(self, x, y):
        self.x = x
        self.y = y

p = Point(2.0, 3.0)
print p.x, p.y
p.x = 2.5
p.z = 42
\end{lstlisting}
\begin{itemize}
\item \texttt{\_\_init\_\_}: Wird automatisch nach Erzeugung eines Objekts aufgerufen
\end{itemize}
\end{frame}

\begin{frame}[fragile]{Methoden auf Objekten}
\begin{lstlisting}[style=Python]
class Point:
    def __init__(self, x, y):
        self.x = x
        self.y = y

    def norm(self):
        n = math.sqrt(self.x**2 + self.y**2)
        return n

p = Point(2.0, 3.0)
print p.x, p.y, p.norm()
\end{lstlisting}
\begin{itemize}
\item Methodenaufruf: automatisch das Objekt als erster Parameter
\item $\rightarrow$ wird "ublicherweise \lstinline{self} genannt
\item\alert{Achtung}: Kein "Uberladen von Methoden m"oglich!
\end{itemize}
\end{frame}

\begin{frame}[fragile]{Objekte in Strings konvertieren}
\onslide<1->
Standard-R"uckgabe von \lstinline{str(...)} f"ur eigene Objekte:
\begin{lstlisting}[style=Shell]
>>> p = Point(2.0, 3.0)
>>> print p  # --> print str(p)
<__main__.Point instance at 0x402d7a8c>
\end{lstlisting}
\vspace{2mm}

\onslide<2->
Das kann man anpassen:
\begin{lstlisting}[style=Python]
def __str__(self):
    return "(%i, %i)" % (self.x, self.y)
\end{lstlisting}
\begin{lstlisting}[style=Shell]
>>> print p
(2, 3)
 \end{lstlisting}
\end{frame}

\begin{frame}[fragile]{Objekte vergleichen}
\onslide<1->
Standard: \texttt{==} pr"uft Objekte eigener Klassen auf Identit"at.
\begin{lstlisting}[style=Shell]
>>> p1 = Point(2.0, 3.0)
>>> p2 = Point(2.0, 3.0)
>>> p1 == p2
False
\end{lstlisting}
%\vspace{2mm}
\onslide<2->
Das kann man anpassen:
\begin{lstlisting}[style=Python]
def __eq__(self, other):
    return (self.x == other.x) and
           (self.y == other.y)
\end{lstlisting}
\begin{lstlisting}[style=Shell]
>>> p1 == p2 # Check for equal values
True
>>> p1 is p2 # Check for identity
False
\end{lstlisting}
\end{frame}

\begin{frame}[fragile]{Objekte vergleichen}
Weitere Vergleichsoperatoren:
\begin{itemize}
\item \texttt{<} : \lstinline{__lt__(self, other)}
\item \texttt{<=} : \lstinline{__le__(self, other)}
\item \texttt{!=} : \lstinline{__ne__(self, other)}
\item \texttt{>} : \lstinline{__gt__(self, other)}
\item \texttt{>=} : \lstinline{__ge__(self, other)}
\end{itemize}
\vspace{2mm}
Alternativ: \lstinline{__cmp__(self, other)}, gibt zur"uck:
\begin{itemize}
\item negativen Integer, wenn \lstinline{self < other}
\item null, wenn \lstinline{self == other}
\item positiven Integer, wenn \lstinline{self > other}
\end{itemize}
\end{frame}

\begin{frame}{Datentypen emulieren}
Man kann mit Klassen vorhandene Datentypen emulieren:
\begin{itemize}
\item Zahlen: Rechenoperationen, \texttt{int(myobj)}, \texttt{float(myobj)}, \dots
\item Funktionen: \texttt{myobj(...)}
\item Sequenzen: \texttt{len(myobj)}, \texttt{myobj[...]}, \lstinline{x in myobj}, ...
\item Iteratoren: \lstinline{for i in myobj}
\end{itemize}
\vspace{2mm}
Siehe dazu Dokumentation:\\
\href{http://docs.python.org/ref/specialnames.html}{http://docs.python.org/ref/specialnames.html}
\end{frame}

\begin{frame}[fragile]{Klassenvariablen}
Haben f"ur alle Objekte einer Klasse stets den gleichen Wert:
\begin{lstlisting}[style=Python]
class Point:
    count = 0  # Count all point objects
    def __init__(self, x, y):
        self.__class__.count += 1
        ...
\end{lstlisting}
\begin{lstlisting}[style=Shell]
>>> p1 = Point(2, 3); p2 = Point(3, 4)
>>> p1.count
2
>>> p2.count
2
>>> Point.count
2
\end{lstlisting}
\end{frame}

\begin{frame}[fragile]{Klassenmethoden und statische Methoden}
\begin{lstlisting}[style=Python]
class Spam:
    spam = "I don't like spam."

    @classmethod
    def cmethod(cls):
        print cls.spam
       
    @staticmethod
    def smethod():
        print "Blah blah."
\end{lstlisting}
\begin{lstlisting}[style=Python]
Spam.cmethod()
Spam.smethod()
s = Spam()
s.cmethod()
s.smethod()
\end{lstlisting}
\end{frame}

\begin{frame}{Vererbung}
Oft hat man verschiedene Klassen, die einander "ahneln.\\
\alert{Vererbung} erlaubt:
\begin{itemize}
\item Hierarchische Klassenstruktur (Ist-ein-Beziehung)
\item Wiederverwenden von "ahnlichem Code
\end{itemize}
\vspace{5mm}
Beispiel: Verschiedene Telefon-Arten
\begin{itemize}
\item Telefon
\item Handy (ist ein Telefon mit zus"atzlichen Funktionen)
\item SmartPhone (ist ein Handy mit zus"atzlichen Funktionen)
\end{itemize}
\end{frame}

\begin{frame}[fragile]{Vererbung}
\begin{lstlisting}[style=Python]
class Phone:
    def call(self):
        pass

class MobilePhone(Phone):
    def send_text(self):
        pass
\end{lstlisting}
Handy erbt jetzt Methoden und Attribute von Telefon.
\begin{lstlisting}[style=Python]
h = MobilePhone()
h.call() # inherited from Phone
h.send_text() # own method
\end{lstlisting}
\end{frame}

\begin{frame}[fragile]{Methoden "uberschreiben}
In der abgeleiteten Klasse k"onnen die Methoden der Elternklasse "uberschrieben werden:
\begin{lstlisting}[style=Python]
class MobilePhone(Phone):
    def call(self):
        find_signal()
        Phone.call(self)
\end{lstlisting}
\end{frame}

\begin{frame}[fragile]{Mehrfachvererbung}
Klassen k"onnen von mehreren Elternklassen erben. Bsp:
\begin{itemize}
\item SmartPhone ist ein Handy
\item SmartPhone ist eine Kamera
\end{itemize}
\begin{lstlisting}[style=Python]
class SmartPhone(MobilePhone, Camera)
    pass

h = SmartPhone()
h.call()  # inherited from MobilePhone
h.take_photo() # inherited from Camera
\end{lstlisting}
Attribute werden in folgender Reihenfolge gesucht:

SmartPhone, MobilePhone, Elterklasse von MobilePhone (rekursiv), Camera, Elternklasse von Camera (rekursiv).
\end{frame}

\begin{frame}[fragile]{Private Attribute}
\begin{itemize}
\item In Python gibt es keine privaten Variablen oder Methoden. 
\item \alert{Konvention:} Attribute, auf die nicht von au"sen zugegriffen werden sollte, beginnen mit einem Unterstrich: \lstinline{_foo}.
\item Um Namenskonflikte zu vermeiden: Namen der Form \lstinline{__foo} werden durch \lstinline{_klassenname__foo} ersetzt:
\end{itemize}
\begin{lstlisting}[style=Python]
class Spam:
    __eggs = 3
\end{lstlisting}
\begin{lstlisting}[style=Shell]
>>> dir(Spam)
>>> ['_Spam__eggs', '__doc__', '__module__']
\end{lstlisting}
\end{frame}

\begin{frame}[fragile]{Properties}
Sollen beim Zugriff auf eine Variable noch Berechnungen oder "Uberpr"ufungen durchgef"uhrt werden: \alert{Getter} und \alert{Setter}
\begin{lstlisting}[style=Python]
class Spam(object):
    def __init__(self):
        self._value = 0
    
    def get_value(self):
        return self._value

    def set_value(self, value):
        if value <= 0:  self._value = 0
        else:  self._value = value

    value = property(get_value, set_value)
\end{lstlisting}
\end{frame}

\begin{frame}[fragile]{Properties}
Auf Properties wird wie auf gew"ohnliche Attribute zugegriffen:
\begin{lstlisting}[style=Shell]
>>> s = Spam()
>>> s.value = 6   # set_value(6)
>>> s.value       # get_value()
>>> 6
>>> s.value = -6  # set_value(-6)
>>> s.value       # get_value()
>>> 0
\end{lstlisting}
\begin{itemize}
\item Getter und Setter k"onnen nachtr"aglich hinzugef"ugt werden, ohne die API zu ver"andern.
\item Zugriff auf \lstinline{_value} immer noch m"oglich
\end{itemize}

\end{frame}


