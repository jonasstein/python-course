\section{Data Types I}

\begin{frame}[fragile]{Numerical Data Types}
\begin{itemize}
\item \alert{\texttt{int}}: corresponds to \texttt{long} in C
\item \alert{\texttt{long}}: unlimited range of values
\item \alert{\texttt{float}}: corresponds to \texttt{double} in C 
\item \alert{\texttt{complex}}: complex numbers
\end{itemize} 
\begin{lstlisting}[style=Python]
a = 1
b = 1L
c = 1.0; c = 1e0
d = 1 + 0j
\end{lstlisting}
\vspace{3mm}
Integers are automatically converted to \texttt{long} if necessary!
\end{frame}

\begin{frame}{Operators on Numbers}
\begin{itemize}
\item \alert{Basic arithmetics}: \texttt{+}, \texttt{-}, \texttt{*}, \texttt{/}
\item Div and modulo operator: \texttt{//}, \hspace{1mm}\texttt{\%}, \hspace{1mm}\texttt{divmod(x, y)}
\item \alert{Absolute value}: \texttt{abs(x)}
\item \alert{Rounding}: \texttt{round(x)}
\item Conversion: \texttt{int(x)}, \texttt{long(x)}, \texttt{float(x)},\\s \texttt{complex(re~[, im=0])}
\item Conjugate of a complex number: \texttt{x.conjugate()}
\item \alert{Power}: \texttt{x ** y}, \hspace{1mm}\texttt{pow(x, y)}
\end{itemize}
Result of a composition of different data types is of the ``bigger'' data type.
\end{frame}

\begin{frame}[fragile]{Strings}
Data type: \alert{\lstinline{str}}
\begin{itemize}
\item \lstinline{s = 'spam'}, \lstinline{s = "spam"}
\item Multiline strings: \lstinline{s = """spam"""}
\item No interpretation of escape sequences: \lstinline{s = r"spam"}
\item Generate strings from other data types: \lstinline{str(1.0)}
\end{itemize}
\begin{lstlisting}[style=Shell]
>>> s = """hello
... world"""
>>> print s
hello
world
>>> print "sp\nam"
sp
am
>>> print r"sp\nam"  # or: print "sp\\nam"
sp\nam
\end{lstlisting}
\end{frame}

\begin{frame}{String Methods}
\begin{itemize}
\item Count appearance of substrings: \lstinline{s.count(sub [, start[, end]])}
\item Begins/ends with a substring? \lstinline{s.startswith(sub[, start[, end]])}, \lstinline{s.endswith(sub[, start[, end]])}
\item All capital/lowercase letters: \lstinline{s.upper()}, \lstinline{s.lower()}
\item Remove whitespace: \lstinline{s.strip([chars])}
\item Split at substring: \lstinline{s.split([sub [,maxsplit]])}
\item Find position of substring: \lstinline{s.index(sub[, start[, end]])}
\item Replace a substring: \lstinline{s.replace(old, new[, count])}
\end{itemize}
More methods: \lstinline{help(str)}, \lstinline{dir(str)}
\end{frame}

\begin{frame}{Lists}
Data type: \alert{\lstinline{list}}
\begin{itemize}
\item \lstinline{s = [1, "spam", 9.0, 42]}, \;\lstinline{s = []}
\item \alert{Append an element}: \lstinline{s.append(x)}
\item Extend with a second list: \lstinline{s.extend(s2)}
\item Count appearance of an element: \lstinline{s.count(x)}
\item Position of an element: \lstinline{s.index(x[, min[, max]])}
\item Insert element at position: \lstinline{s.insert(i, x)}
\item Remove and return element at position: \lstinline{\s.pop([i])}
\item \alert{Remove element}: \lstinline{s.remove(x)}
\item Reverse list: \lstinline{s.reverse()}
\item \alert{Sort}: \lstinline{s.sort([cmp[, key[, reverse]]])}
\item Sum of the elements: \lstinline{sum(s)}
\end{itemize}
\end{frame}

\begin{frame}{Operations on Sequences}
Strings and lists have much in common: They are \alert{sequences}.
\begin{itemize}
\item Does/doesn't s contain an element?\\
 \lstinline{x in s}, \lstinline{x not in s}
\item Concatenate sequences: \lstinline{s + t}
\item Multiply sequences: \lstinline{n * s}, \lstinline{s * n}
\item \alert{i-th element}: \lstinline{s[i]}, \alert{i-th to last  element}: \lstinline{s[-i]}
\item Subsequence: \lstinline{s[i:j]}, with step size k: \lstinline{s[i:j:k]}
\item Subsequence from beginning/to end: \lstinline{s[:-i]}, \lstinline{s[i:]}, \lstinline{s[:]}
\item \alert{Length}: \lstinline{len(s)}
\item smallest/largest element: \lstinline{min(s)}, \lstinline{max(s)}
\item Assignments: \lstinline{(a, b, c) = s} \\
$\rightarrow$ \lstinline{a = s[0]}, \lstinline{b = s[1]}, \lstinline{c = s[2]}
\end{itemize}
\end{frame}

\begin{frame}[fragile]{Sequencen}
\begin{itemize}
\item Another sequence: data type \alert{\texttt{tuple}}: \lstinline{a = (1, 2.0, "3")}
\item List are mutable
\item Strings and tuples are immutable
\begin{itemize}
\item No assignment \lstinline{s[i] = ...}
\item No appending and removing of elements
\item Functions like \texttt{upper} return a new string!
\end{itemize}
\end{itemize}
\begin{lstlisting}[style=Shell]
>>> s1 = "spam"
>>> s2 = s1.upper()
>>> s1
'spam'
>>> s2
'SPAM'
\end{lstlisting}
\end{frame}

\begin{frame}[fragile]{References}
\begin{itemize}
\item In Python, everything is a reference to an object!
\item Careful with assignments:
\end{itemize}
\begin{lstlisting}[style=Shell]
>>> s1 = [1, 2, 3, 4]
>>> s2 = s1
>>> s2[1] = 17
>>> s1
[1, 17, 3, 4]
>>> s2
[1, 17, 3, 4]
\end{lstlisting}
Flat copy of a list: \lstinline{s2 = s1[:]} or \lstinline{s2 = list(s1)}
\end{frame}

\begin{frame}[fragile]{Boolean Values}
Data type \alert{bool}: \texttt{True}, \texttt{False}

Values that are evulated to \texttt{False}:
\begin{itemize}
\item \texttt{None}
\item \texttt{False}
\item \texttt{0} (in every numerical data type)
\item empty strings, lists and tuples: \texttt{''}, \texttt{[]}, \texttt{()}
\item empty dictionaries: \texttt{\{\}}
\item empty sets
\end{itemize}
All other Objects of built-in data types are evaluated to \texttt{True}!
\begin{lstlisting}[style=Shell]
>>> bool([1, 2, 3])
True
>>> bool("")
False
\end{lstlisting}
\end{frame}

