\section{Datentypen I}

\begin{frame}[fragile]{Numerische Datentypen}
\begin{itemize}
\item \alert{\texttt{int}}: entspricht \texttt{long} in C
\item \alert{\texttt{long}}: unbegrenzter Wertebereich
\item \alert{\texttt{float}}: enspricht \texttt{double} in C 
\item \alert{\texttt{complex}}: komplexe Zahlen
\end{itemize} 
\begin{lstlisting}[style=Python]
a = 1
b = 1L
c = 1.0; c = 1e0
d = 1 + 0j
\end{lstlisting}
\vspace{3mm}
Integers werden bei Bedarf automatisch in long umgewandelt! 
\end{frame}

\begin{frame}{Operatoren auf Zahlen}
\begin{itemize}
\item \alert{Grundrechenarten}: \texttt{+}, \texttt{-}, \texttt{*}, \texttt{/}
\item Div- und Modulo-Operator: \texttt{//}, \hspace{1mm}\texttt{\%}, \hspace{1mm}\texttt{divmod(x, y)}
\item \alert{Betrag}: \texttt{abs(x)}
\item \alert{Runden}: \texttt{round(x)}
\item Konvertierung: \texttt{int(x)}, \texttt{long(x)}, \texttt{float(x)}, \texttt{complex(re~[, im=0])}
\item Konjugierte einer komplexen Zahl: \texttt{x.conjugate()}
\item \alert{Potenzen}: \texttt{x ** y}, \hspace{1mm}\texttt{pow(x, y)}
\end{itemize}
Ergebnis einer Verkn"upfung unterschiedlicher Datentypen ist vom Typ des \glqq gr"o"seren\grqq{} Datentyps.
\end{frame}

\begin{frame}[fragile]{Strings}
Datentyp: \alert{\lstinline{str}}
\begin{itemize}
\item \lstinline{s = 'spam'}, \lstinline{s = "spam"}
\item Mehrzeilige Strings: \lstinline{s = """spam"""}
\item keine Interpretation von Escape-Sequenzen: \lstinline{s = r"spam"}
\item Strings aus anderen Datentypen erzeugen: \lstinline{str(1.0)}
\end{itemize}
\begin{lstlisting}[style=Shell]
>>> s = """hallo
... welt"""
>>> print s
hallo
welt
>>> print "sp\nam"
sp
am
>>> print r"sp\nam"  # or: print "sp\\namam"
sp\nam
\end{lstlisting}
\end{frame}

\begin{frame}{String-Methoden}
\begin{itemize}
\item Vorkommen von Substrings z"ahlen: \lstinline{s.count(sub [, start[, end]])}
\item beginnt/endet s mit einem Substring? \lstinline{s.startswith(sub[, start[, end]])}, \lstinline{s.endswith(sub[, start[, end]])}
\item s in Gro"s-/Kleinbuchstaben: \lstinline{s.upper()}, \lstinline{s.lower()}
\item Leerraum entfernen: \lstinline{s.strip([chars])}
\item an Substrings trennen: \lstinline{s.split([sub [,maxsplit]])}
\item Position eines Substrings finden: \lstinline{s.index(sub[, start[, end]])}
\item einen Substring ersetzen: \lstinline{s.replace(old, new[, count])}
\end{itemize}
Weitere Methoden: \lstinline{help(str)}, \lstinline{dir(str)}
\end{frame}

\begin{frame}{Listen}
Datentyp: \alert{\lstinline{list}}
\begin{itemize}
\item \lstinline{s = [1, "spam", 9.0, 42]}, \;\lstinline{s = []}
\item \alert{Element anh"angen}: \lstinline{s.append(x)}
\item um zweite Liste erweitern: \lstinline{s.extend(s2)}
\item Vorkommen eines Elements z"ahlen: \lstinline{s.count(x)}
\item Position eines Elements: \lstinline{s.index(x[, min[, max]])}
\item Element an Position einf"ugen: \lstinline{s.insert(i, x)}
\item Element an Position l"oschen und zur"uckgeben: \lstinline{\s.pop([i])}
\item \alert{Element l"oschen}: \lstinline{s.remove(x)}
\item Liste umkehren: \lstinline{s.reverse()}
\item \alert{Sortieren}: \lstinline{s.sort([cmp[, key[, reverse]]])}
\item Summe der Elemente: \lstinline{sum(s)}
\end{itemize}
\end{frame}

\begin{frame}{Operationen auf Sequenzen}
Strings und Listen haben viel gemeinsam: Sie sind \alert{Sequenzen}.
\begin{itemize}
\item Ist ein Element in s enhalten/nicht enthalten?\\
 \lstinline{x in s}, \lstinline{x not in s}
\item Sequenzen aneinanderh"angen: \lstinline{s + t}
\item Sequenzen vervielf"altigen: \lstinline{n * s}, \lstinline{s * n}
\item \alert{i-tes Element}: \lstinline{s[i]}, von hinten: \lstinline{s[-i]}
\item Subsequenz: \lstinline{s[i:j]}, mit Schrittweite k: \lstinline{s[i:j:k]}
\item Subsequenz von Anfgang/bis Ende: \lstinline{s[:-i]}, \lstinline{s[i:]}, \lstinline{s[:]}
\item \alert{L"ange}: \lstinline{len(s)}
\item kleinstes/gr"o"stes Element: \lstinline{min(s)}, \lstinline{max(s)}
\item Zuweisungen: \lstinline{(a, b, c) = s} \\
$\rightarrow$ \lstinline{a = s[0]}, \lstinline{b = s[1]}, \lstinline{c = s[2]}
\end{itemize}
\end{frame}

\begin{frame}[fragile]{Sequenzen}
\begin{itemize}
\item Auch eine Sequenz: Datentyp \alert{\texttt{tuple}}: \lstinline{a = (1, 2.0, "3")}
\item Listen sind ver"anderbar
\item Strings und Tupel sind nicht ver"anderbar
\begin{itemize}
\item Keine Zuweisung \lstinline{s[i] = ...}
\item Kein Anh"angen und L"oschen von Elementen
\item Funktionen wie \texttt{upper} liefern einen neuen String zur"uck!
\end{itemize}
\end{itemize}
\begin{lstlisting}[style=Shell]
>>> s1 = "spam"
>>> s2 = s1.upper()
>>> s1
'spam'
>>> s2
'SPAM'
\end{lstlisting}
\end{frame}

\begin{frame}[fragile]{Referenzen}
\begin{itemize}
\item In Python ist alles eine Referenz auf ein Objekt!
\item Vorsicht bei Zuweisungen:
\end{itemize}
\begin{lstlisting}[style=Shell]
>>> s1 = [1, 2, 3, 4]
>>> s2 = s1
>>> s2[1] = 17
>>> s1
[1, 17, 3, 4]
>>> s2
[1, 17, 3, 4]
\end{lstlisting}
Flache Kopie einer Liste: \lstinline{s2 = s1[:]} oder \lstinline{s2 = list(s1)}
\end{frame}

\begin{frame}[fragile]{Wahrheitswerte}
Datentyp \alert{bool}: \texttt{True}, \texttt{False}

Werte, die zu \texttt{False} ausgewertet werden:
\begin{itemize}
\item \texttt{None}
\item \texttt{False}
\item \texttt{0} (in jedem numerischen Datentyp)
\item leere Strings, Listen und Tupel: \texttt{''}, \texttt{()}, \texttt{[]}
\item leere Dictionaries: \texttt{\{\}}
\item leere Sets
\end{itemize}
Andere Objekte von eingebauten Datentypen werden stets zu \texttt{True} ausgewertet!
\begin{lstlisting}[style=Shell]
>>> bool([1, 2, 3])
True
>>> bool("")
False
\end{lstlisting}
\end{frame}

