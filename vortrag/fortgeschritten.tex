\section{Fortgeschrittene Techniken}

\begin{frame}[fragile]{Funktionsparameter aus Listen und Dictionaries}
\begin{lstlisting}
def spam(a, b, c, d):
    print a, b, c, d
\end{lstlisting}
Man kann positionale Parameter aus Listen erzeugen:
\begin{lstlisting}[style=Shell]
>>> args = [3, 6, 2, 3]
>>> spam(*args)
3 6 2 3
\end{lstlisting}
Man kann Keyword-Paramter aus Dictionaries erzeugen:
\begin{lstlisting}[style=Shell]
>>> kwargs = {"c": 5, "a": 2, "b": 4, "d":1}
>>> spam(**kwargs)
2 4 5 1
\end{lstlisting}
\end{frame}

\begin{frame}[fragile]{Funktionen mit beliebigen Parametern}
\begin{lstlisting}
def spam(*args, **kwargs):
    for i in args:
        print i
    for i in kwargs:
        print i, kwargs[i]
\end{lstlisting}
\begin{lstlisting}[style=Shell]
>>> spam(1, 2, c=3, d=4)
1
2
c 3
d 4
\end{lstlisting}
\end{frame}


\begin{frame}[fragile]{List Comprehension}
Abk"urzende Schreibweise zum Erstellen von Listen aus for-Schleifen. Statt:
\begin{lstlisting}
a = []
for i in range(10):
    a.append(i**2)
\end{lstlisting}
kann man schreiben:
\begin{lstlisting}
a = [i**2 for i in range(10)]
\end{lstlisting}
\vspace{2mm}
Mit Bedingung:
\begin{lstlisting}
a = [i**2 for i in range(10) if i != 4]
\end{lstlisting}
\end{frame}

\begin{frame}[fragile]{Anonyme Funktionen: Lambda}
\begin{lstlisting}[style=Shell]
>>> f = lambda x, y: x + y
>>> f(2, 3)
4
>>> (lambda x: x**2)(3)
9
\end{lstlisting}
N"utzlich, wenn einfache Funktionen als Parameter "ubergeben werden sollen.
\begin{lstlisting}[style=Python]
l = ["alice", "Bob"]
l.sort()
l.sort(lambda a,b: cmp(a.upper(), b.upper()))
\end{lstlisting}

\end{frame}

\begin{frame}[fragile]{Map}
Anwenden einer Funktion auf alle Elemente einer Liste:
\begin{lstlisting}[style=Shell]
>>> l = [1, 4, 81, 9]
>>> map(math.sqrt, l)
[1.0, 2.0, 9.0, 3.0]
>>> map(lambda x: x * 2, l)
[2, 8, 162, 18]
\end{lstlisting}
Wenn die Funktion mehr als einen Parameter nimmt, kann je zus"atzlichem Parameter eine weitere Liste "ubergeben werden:
\begin{lstlisting}[style=Shell]
>>> map(math.pow, a, [1, 2, 3, 4])
[1.0, 16.0, 531441.0, 6561.0]
\end{lstlisting}
\end{frame} 

\begin{frame}[fragile]{Filter}
Wie Map, jedoch enth"alt die Egebnisliste nur die Elemente, welche wahr sind:
\begin{lstlisting}[style=Shell]
>>> l = [1, 2, 3, 4, 5, 6, 7, 8, 9]
>>> filter(lambda x: x % 2, l)
[1, 3, 5, 7, 9]
\end{lstlisting}
\end{frame} 

\begin{frame}[fragile]{Zip}
Zusammenf"ugen mehrer Sequenzen zu einer Liste von Tupeln:
\begin{lstlisting}[style=Shell]
>>> zip("ABC", "123")
[('A', '1'), ('B', '2'), ('C', '3')]
>>> zip([1, 2, 3], "ABC", "XYZ")
[(1, 'A', 'X'), (2, 'B', 'Y'), (3, 'C', 'Z')]
\end{lstlisting}
\end{frame}


\begin{frame}[fragile]{defaultdict}
Ein Dictionary, welches Defaultwerte f"ur nicht vorhandene Schl"usselw"orter erzeugen kann:
\begin{lstlisting}
from collections import defaultdict
haeufigkeiten = defaultdict(lambda: 0)

for c in "Hallo Welt!":
    haeufigkeiten[c] += 1
\end{lstlisting}
Paramter (optional): Funktion zum Erzeugen der Defaultwerte
\end{frame}

\begin{frame}[fragile]{Iteratoren}
Was passiert, wenn \lstinline{for} auf einem Objekt aufgerufen wird?
\begin{lstlisting}[style=Python]
for i in obj:
    pass
\end{lstlisting}
\begin{itemize}
\item Auf \lstinline{obj} wird die \lstinline{__iter__}-Methode aufgerufen, welche einen \alert{Iterator} zur"uckgibt
\item Auf dem Iterator wird bei jedem Durchlauf \alert{\lstinline{next()}} aufgerufen
\item Eine \lstinline{StopIteration}-Ausnahme beendet die for-Schleife
\end{itemize}
\end{frame}

\begin{frame}[fragile]{Iteratoren}
\begin{lstlisting}[style=Python]
class Reverse:
    def __init__(self, data):
        self.data = data
        self.index = len(data)
    def __iter__(self):
        return self
    def next(self):
        if self.index == 0:
            raise StopIteration
        self.index = self.index - 1
        return self.data[self.index]
\end{lstlisting}
\begin{lstlisting}[style=Shell]
>>> for char in Reverse("spam"):
...     print char,
...
m a p s
\end{lstlisting}
\end{frame}

\begin{frame}[fragile]{Generatoren}
Einfache Weise, Iteratoren zu erzeugen:
\begin{itemize}
\item Werden wie Funktionen definiert
\item \lstinline{yield}-Statement, um Daten zur"uckzugeben und beim n"achsten \lstinline{next}-Aufruf dort weiterzumachen
\end{itemize}
\begin{lstlisting}[style=Python]
def reverse(data):
    for element in data[::-1]:
        yield element
\end{lstlisting}
\begin{lstlisting}[style=Shell]
>>> for char in reverse("spam"):
...     print char,
...
m a p s
\end{lstlisting}
\end{frame}

\begin{frame}[fragile]{Generator-Audr"ucke}
"Ahnlich zu List Comprehensions kann man anonyme Iteratoren erzeugen:
\begin{lstlisting}[style=Shell]
>>> data = "spam"
>>> for c in (elem for elem in data[::-1]):
...     print c,
...
m a p s
\end{lstlisting}
\end{frame}


\begin{frame}[fragile]{Dynamische Attribute}
Erinnerung: Man kann Attribute von Objekten zur Laufzeit hinzuf"ugen:
\begin{lstlisting}
class Empty:
    pass

a = Empty()
a.spam = 42
a.eggs = 17
\end{lstlisting}
\vspace{2mm}
Und entfernen:
\begin{lstlisting}
del a.spam
\end{lstlisting}
\end{frame}

\begin{frame}[fragile]{getattr, setattr}
Man kann Attribute von Objekten als Strings ansprechen:
\begin{lstlisting}
import math
f = getattr(math, "sin")
print f(x) # sin(x)
\end{lstlisting}
\vspace{2mm}
\begin{lstlisting}
a = Empty()
setattr(a, "spam", 42)
print a.spam
\end{lstlisting}
N"utzlich, wenn man z.B. Attributnamen aus User-Input oder Dateien liest.
\end{frame}

